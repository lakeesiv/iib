\newcommand{\name}{Lakee Sivaraya}
\newcommand{\crsid}{ls914}
\newcommand{\labgroup}{46}
\newcommand{\college}{Emmanuel College}
\newcommand{\logodir}{./logo/Emmanuel.pdf}

\newcommand{\tikzcolor}{black}
\newcommand{\tikzcoloropposite}{white}

\usepackage{anyfontsize}
% \usepackage{cancel} # for strike-through text
\usepackage{lipsum}
\usepackage{graphicx}
\usepackage{enumerate}
% \usepackage{romannum}
\usepackage{amsmath}
\usepackage{amssymb}
\usepackage{mathtools}
\usepackage[table,xcdraw]{xcolor}
\renewcommand{\arraystretch}{1.2}
\usepackage{hyperref}
\hypersetup{hidelinks}
% \usepackage{subfig}
\usepackage{subcaption}
\captionsetup{compatibility=false}
\usepackage{placeins}
\usepackage{booktabs}
\usepackage{caption}
\usepackage{relsize} 
\usepackage{xcolor}
% \usepackage[utf8]{inputenc}
\usepackage[a4paper, total={7in, 10in}]{geometry}
\usepackage{siunitx}
% \usepackage[utf8]{inputenc}
\usepackage{pythonhighlight}
\usepackage[utf8]{inputenc}
\usepackage[title, toc]{appendix} 
\numberwithin{figure}{section}
\numberwithin{table}{section}
\usepackage{algorithm}
\usepackage{algpseudocode}
\usepackage{float}
\usepackage{multirow}
\usepackage{tabularx}
\usepackage{multicol}
\usepackage{bm}
\usepackage[
backend=biber,
style=alphabetic,
]{biblatex}


% \usepackage{tikz}
% \usepackage{pgfplots}
% \usepgfplotslibrary{groupplots,dateplot}
% \usetikzlibrary{patterns,shapes.arrows}
% \pgfplotsset{compat=newest} % Allows to place the legend below plot
% \usepgfplotslibrary{units} % Allows to enter the units nicely
% \sisetup{
%   round-mode          = places,
%   round-precision     = 2,
% }
% \usetikzlibrary{patterns,arrows}
% \pgfplotsset{width=10cm,compat=1.9}
% \pgfplotsset{
%     dirac/.style={
%         mark=triangle*,
%         mark options={scale=2},
%         ycomb,
%         scatter,
%         visualization depends on={y/abs(y)-1 \as \sign},
%         scatter/@pre marker code/.code={\scope[rotate=90*\sign,yshift=-2pt]}
%     }
% }



\DeclareUnicodeCharacter{2212}{−}


\hypersetup{
	colorlinks   = true, %Colours links instead of ugly boxes
	urlcolor     = red, %Colour for external hyperlinks
	linkcolor    = gray, %Colour of internal links
	citecolor   = red %Colour of citations
}


\usepackage{listings}
\usepackage{xcolor}
\pagenumbering{arabic}

\definecolor{codegreen}{rgb}{0,0.6,0}
\definecolor{codegray}{rgb}{0.5,0.5,0.5}
\definecolor{codepurple}{rgb}{0.58,0,0.82}
\definecolor{backcolour}{rgb}{0.95,0.95,0.92}

\lstdefinestyle{mystyle}{
    backgroundcolor=\color{backcolour},   
    commentstyle=\color{codegreen},
    keywordstyle=\color{magenta},
    numberstyle=\tiny\color{codegray},
    stringstyle=\color{codepurple},
    basicstyle=\ttfamily\footnotesize,
    keywordstyle = [2]{\color{blue}},
    keywordstyle = [3]{\color{blue}},
    breakatwhitespace=false,         
    breaklines=true,                 
    captionpos=b,                    
    keepspaces=true,                 
    numbers=left,                    
    numbersep=5pt,                  
    showspaces=false,                
    showstringspaces=false,
    showtabs=false,                  
    tabsize=2,
    morecomment=[s]{\#}{\^^M},
    otherkeywords = {;},
    morekeywords = [2]{;},
}

\lstset{style=mystyle}

\newcommand{\createtitlepage}{
	
	\title{\textbf{\fontsize{24.88}{12}\selectfont \lrtitle}}
	\author{{\LARGE \name (ls914)}}
	
	\begin{titlepage}
		\maketitle
		\centering
		\vfill
		{\bfseries\Large
			\college\vspace{0.4in}
		}    
		    
		\includegraphics[width=3cm]{\logodir} 
		\vfill
		\vfill
		\thispagestyle{empty}
		\newpage
		
	\end{titlepage}
}

\newcommand{\createnotestitlepage}{
	
	\title{\textbf{\fontsize{24.88}{12}\selectfont \lrtitle}}
	\author{{\LARGE \name \space (ls914)}}
	
	\begin{titlepage}
		\maketitle

		\centering
		\vfill
		{\bfseries\Large
			\college\vspace{0.4in}
		}    
		    
		\includegraphics[width=3cm]{\logodir} 
		\vfill
		\vfill
		\thispagestyle{empty}
		{\hypersetup{linkcolor=gray}
		\tableofcontents
		}
		\newpage
		
	\end{titlepage}
}


\newcommand{\TOC}[1]{
	{\hypersetup{linkcolor=#1}
		\tableofcontents
	}
	\newpage
	
}

\newcommand{\darkmode}{
	\pagecolor[rgb]{0.12156862745
		,0.12156862745
		,0.12156862745
	} 
	
	\color[rgb]{.7,.7,.7} 
	\renewcommand{\tikzcolor}{gray}
	\renewcommand{\tikzcoloropposite}{black}


}


% \newcommand{\num}[1]{\begin{enumerate}
% 	#1
% 	\end{enumerate}}


\newcommand{\fontsz}[1]{
	\fontsize{#1}{#1}
}
\newcommand{\noi}{\noindent}

\newcommand{\tikzenv}[2]{
\begin{center}
\begin{circuitikz}[scale=#1, transform shape]
#2
\end{circuitikz}
\end{center}
}

